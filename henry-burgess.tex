% Author: Henry Burgess
\documentclass{article}

% Setup packages
\usepackage{hyperref}
\usepackage{xcolor}
\usepackage{geometry}
\usepackage{verbatim}
\geometry{letterpaper, portrait, margin=0.5in}

% Paragraph spacing
\setlength{\parindent}{0pt}
\setlength{\parskip}{2pt}

\begin{document}
  %  Header section with name and contact details
  \begin{center}
    {\huge\textbf{\uppercase{Henry Burgess}}}

    St. Louis, MO, United States

    henryjburg@gmail.com, +1 (314) 891-2285
  \end{center}

  \smallbreak

  % Introduction Statement
  % \begin{comment}
  Generalist Software Engineer with expertise bridging hardware and software development to create innovative tools for translational neuroscience research. Proven track record delivering maintainable solutions and leading cross-functional technical projects.
  % \end{comment}

  % Education
  \section*{\centering\uppercase{Education}}
  
  {\textbf{\uppercase{Master of Engineering Management}}}

  \textbf{Washington University in St. Louis} \hfill \textit{January 2025 - August 2026}
  
  \textit{St. Louis, MO, United States}

  \textbf{GPA:} 4.0

  \smallbreak

  {\textbf{\uppercase{Bachelor of Engineering (Hons.) (Software)}}}

  \textbf{The University of Queensland} \hfill \textit{January 2017 - June 2021}

  \textit{Brisbane, Australia}

  \textbf{GPA:} 3.85, Honors Class I

  \textbf{Thesis:} ``Implementation of Online Neuropsychological Tasks using JavaScript"

  \textbf{Thesis Supervision:} Linda Richards AO, FAA, FAHMS, PhD; Ryan Dean, PhD; Richard Thomas, MAppSc

  \textbf{Awards and Honors:} UQ Future Leader (Class of 2021), Hawken Scholar (2020), Dean's commendation for academic excellence (2019, 2021)

  \smallbreak

  % Professional Experience
  \section*{\centering\uppercase{Professional Experience}}

  {\textbf{\uppercase{Software Engineer II \hfill Washington University School of Medicine in St. Louis}}}

  \textit{St. Louis, MO, United States \hfill September 2021 - Present}
  \begin{itemize}
    \itemsep-0.2em
    \item Ownership of the full software development lifecycle for various research projects, ensuring timely delivery of software solutions critical to neuroscientific research.
    \item Implemented and deployed web-based behavioral experiments (TypeScript, React, WebR, Python), achieving complete data collection from 700+ global participants, supporting 3 peer-reviewed publications.
    \item Engineered existing behavioral experiments for delivery via VR platforms using Unity and C\#. Developed VR monitoring utilities using Python and WebSocket communication for wireless headset monitoring and experiment control.
    \item Architected and deployed a full-stack data management platform (TypeScript, React, MongoDB, Node.js) enabling rapid management and search across hundreds of scientific metadata records. Awarded Virtual Institute for Scientific Software partnership with Georgia Tech, evaluated for licensing with WashU’s Office of Technology Management.
    \item Designed and implemented a Python-based remote-control panel and device software for a Raspberry Pi-based mice behavior apparatus, enabling preliminary data collection for PhD students and future funding applications.
    \item Presented work at major industry conferences (Society for Neuroscience, US Research Software Engineers Association).
    \item Conducted a project management review of 16-member lab, analyzing 2,500+ historical work items using an LLM-based pipeline, delivering targeted and actionable insights to propose new workflows.
  \end{itemize}
  
  \smallbreak

  {\textbf{\uppercase{Teaching Assistant \hfill The University of Queensland}}}

  \textit{Brisbane, Australia \hfill January 2019 - June 2022}
  \begin{itemize}
    \itemsep-0.2em
    \item Communicated advanced concepts, guided student learning, and collaborated on coursework development.
    \item Mentored students in Agile methodologies by overseeing collaborative Java projects using GitHub.
    \item Introduced students to Python, Object-Oriented Programming, and the Software Development Life Cycle.
  \end{itemize}

  \smallbreak

  {\textbf{\uppercase{Research Assistant \hfill Queensland Brain Institute}}}

  \textit{Brisbane, Australia \hfill January 2021 - September 2021}
  \begin{itemize}
    \itemsep-0.2em
    \item Delivered multiple behavioral research tasks using JavaScript, resulting in continued research data collection and scholarly publications.
    \item Assisted with data collection and behavioral testing of research participants.
  \end{itemize}

  \pagebreak
  
  \section*{\centering\uppercase{Projects}}

  \textbf{Metadatify}

  \url{https://github.com/Brain-Development-and-Disorders-Lab/mars}

  An open-source web application for metadata management that encourages FAIR data principles by making lab-generated metadata searchable and accessible.
  
  \smallbreak
  
  \textbf{Intentions Game}

  \url{https://github.com/Brain-Development-and-Disorders-Lab/task_intentions}

  Online social value orientation (SVO) task built using React and WebR, allowing execution of a R-based computational model in-browser to provide dynamic responses to participant inputs. Collected data from 500+ global participants across multiple adolescent and adult studies.
  
  \smallbreak

  \textbf{task\_vr\_rdk}

  \url{https://github.com/Brain-Development-and-Disorders-Lab/task_vr_rdk}

  VR Random Dot Kinematogram (RDK) task adapted from Bang et al. (2018) that incorporates eye-tracking functionality and lateralized visual presentation to examine interhemispheric integration.
  
  \smallbreak
  
  \textbf{headsup}

  \url{https://github.com/Brain-Development-and-Disorders-Lab/headsup}

  VR experiment monitoring tool allowing researchers to operate and observe VR experiments developed using Unity over a wireless network via a control panel.

  % Publications
  \section*{\centering\uppercase{Publications}}

  {\textbf{\uppercase{Peer-Reviewed and Preprint}}}
  
  \begin{itemize}
    \itemsep-0.2em
    \item Barnby, J. M., Nguyen, J., Griem, J., Wloszek, M., \textbf{Burgess, H.}, Richards, L., ... \& Fonagy, P. (2024). Self-Other Generalisation Shapes Social Interaction and Is Disrupted in Borderline Personality Disorder. \url{https://doi.org/10.31234/osf.io/kcwm8}
    \item Kingston, J. L., Ellett, L., \textbf{Burgess, H.}, Richards, L., Barnby, J.M. (2025). Social exclusion increases paranoia and reduces self and other learning flexibility in adolescents. OSF. \url{https://doi.org/10.31234/osf.io/pqn6h_v2}
    \item Richards, L. J., Barnby, J., Dean, R., \textbf{Burgess, H.}, Kim, J., Teunisse, A., ... \& Dayan, P. (2021). Increased persuadability and credulity in people with corpus callosum dysgenesis. Cortex. \url{https://doi.org/10.1016/j.cortex.2022.07.009}
  \end{itemize}

  \smallbreak

  {\textbf{\uppercase{Conference Abstracts}}}

  \begin{itemize}
    \itemsep-0.2em
    \item \textit{“MARS: An Open-Source Application for Managing and Searching Scientific Metadata”}, United States Research Software Engineer Association Annual Conference, October 2024
    \item \textit{“Exploring the Use of Virtual Reality Experiences in Research Participation in Behavioral Data Collection”}, International Research Consortium for the Corpus Callosum and Cerebral Connectivity, June 2024
    \item \textit{“Realizing Dynamic Cognitive Tasks with Cloud-based Computation”}, United States Research Software Engineer Association Annual Conference, October 2023
    \item \textit{“Realizing Dynamic Cognitive Tasks with Cloud-based Computation”}, Cognitive Neuroscience Society Meeting, March 2023
    \item \textit{“Neurocog.js; A new tool for running cognitive experiments in both lab and online environments.”}, Society for Neuroscience Meeting, October 2022
    \item \textit{“Enabling behavioral research with Computer Science”}, International Research Consortium for the Corpus Callosum and Cerebral Connectivity, July 2022
  \end{itemize}
  
  {\textbf{\uppercase{Talks}}}

  \begin{itemize}
    \itemsep-0.2em
    \item \textit{“Personal Project Management: Principles for building a personal project management workflow as a Research Software Engineer”}, United States Research Software Engineer Association Annual Conference, October 2025
  \end{itemize}

  % Memberships
  \section*{\centering\uppercase{Memberships}}

  {\textbf{The United States Research Software Engineer Association (US-RSE)}}

  \textit{Member \hfill 2022 - Present}

  \smallbreak

  {\textbf{International Research Consortium for the Corpus Callosum and Cerebral Connectivity (IRC\textsuperscript{5})}}

  \textit{Associate member, Neuropsychology \hfill 2021 - Present}

  \smallbreak

  {\textbf{Society for Neuroscience}}

  \textit{Regular Member \hfill 2022 - 2023}

  \smallbreak

  {\textbf{Cognitive Neuroscience Society}}

  \textit{Graduate Member \hfill 2022 - 2023}

  \smallbreak

  {\textbf{Engineers Australia}}

  \textit{Graduate Member \hfill 2021 - 2024}

  % Memberships
  \section*{\centering\uppercase{Leadership and Service}}

  {\textbf{The United States Research Software Engineer Association (US-RSE)}}

  \textit{Mentor \hfill 2025 - Present}

  \textit{Coordinator, St. Louis Metro Regional Affinity Group \hfill 2026 - Present}

\end{document}
